\documentclass[useAMS, usenatbib, referee]{biom}
%\documentclass[useAMS, usenatbib]{biom}
\usepackage{appendix, subfigure}
\usepackage{framed}
%\usepackage{authblk}
\usepackage{bm,amsmath}
\usepackage{amsfonts}
%\usepackage{hyperref}
\usepackage{blkarray}
\usepackage[colorinlistoftodos,linecolor=gray,backgroundcolor=white]{todonotes}

% my command definitions
%\newtheorem{exercise}{Exercise}[chapter]
%\newtheorem{example}{Example}[chapter]
\newcommand{\ol}[1]{\overline{#1}}
\newcommand{\D}{\displaystyle}
\newcommand{\T}{\textstyle}
\newcommand{\s}{\scriptstyle}
\newcommand{\mc}[1]{\mathcal{#1}}
\newcommand{\bex}{\begin{exercise}}
\newcommand{\eex}{\end{exercise}}
\newcommand{\beg}{\begin{example}}
\newcommand{\eeg}{\end{example}}
\newcommand{\schist}[1]{\mbox{\tiny{#1}}}
\newcommand{\chist}[1]{\mbox{\scriptsize{#1}}}
\newcommand{\agemo}[1]{\mbox{$\underline{\omega}_{\/\downarrow #1}$}}
\newcommand{\ul}[1]{\mbox{$\underline{#1}$}}
\newcommand{\wh}[1]{\mbox{$\widehat{#1}$}}
\newcommand{\bv}{\begin{verbatim}}
\newcommand{\ev}{\end{verbatim}}
\newcommand{\bi}{\begin{itemize}}
\newcommand{\ei}{\end{itemize}}
\newcommand{\bfig}{\begin{figure}}
\newcommand{\efig}{\end{figure}}
\newcommand{\pref}[1]{\protect\ref{#1}}
\newcommand{\plab}[1]{\protect\label{#1}}
\newcommand{\be}{\begin{eqnarray}}
\newcommand{\bes}{\begin{eqnarray*}}
\newcommand{\ee}{\end{eqnarray}}
\newcommand{\ees}{\end{eqnarray*}}
\newcommand{\bo}[1]{\bf #1}
\newcommand{\bmt}[1]{\mbox{\boldmath $#1$}}
\newcommand{\tm}[1]{\mbox{\tiny{$#1$}}}
\newcommand{\vc}[2]{\mbox{$ \left[ \begin{array}{c} #1\\ \vdots \\ #2 
 \end{array} \right] $}}
\newcommand{\mt}[4]{\mbox{$ \left[ \begin{array}{c,c,c} #1 & \cdots & #2\\ 
 \vdots & \ddots & \vdots\\ #3 & \cdots & #4 \end{array} \right] $}}

% RMF: added this new command \dotomega which can be redefined to \dot{\omega} or just \omega if you don't like it:
\newcommand{\dotomega}{\tilde{\omega}}

\begin{document}



\begin{centering}
{\huge Supporting Information for \\ A latent capture history model for digital aerial surveys} \\
{\normalsize by} \\
{\Large D. L. Borchers, P. Nightingale, B. C. Stevenson, and R. M. Fewster.} \\
\end{centering}



\section{Web Appendix A: Derivation of $f_{T}(t)$}
\label{appx:firstpassage}

%\todo[inline]{RMF - I've made several edits to tidy up the derivation here.}
Define the time and forward coordinate at which observer 1 passes over an animal to be 0. The animal's forward coordinate at time $t$ is $\sigma W_t$, where $W_t$ is a one-dimensional Brownian motion. The forward coordinate of observer 2 at time $t$ is $-vl+vt$. The time at which observer 2 passes over the animal is therefore the minimum $t$ such that
\be
-vl+vt&=&\sigma W_t \;\;\;\Rightarrow \;\;\;\frac{vt}{\sigma} - W_t \;=\; \frac{vl}{\sigma}\,.
\ee
\noindent
The passage time for observer 2 is therefore $T=\inf\{t: vt/\sigma + B_t = vl/\sigma\}$, where $B_t=-W_t$ is also a Brownian motion. Now if a particle follows Brownian motion with drift parameter $c$, such that its location at time $t$ is $X_t=ct+B_t$, then the random variable $T=\inf\{t: X_t=a\}$ is the first passage time to location $a$, and has probability density function $f_{T} (t) = a \exp \Big\{ \frac{- (a-ct)^2}{2t} \Big\} / (\sqrt{2 \pi t^3})$. Substituting $c=v/\sigma$ and $a=vl/\sigma$, we obtain the probability density of the time $T$ at which observer 2 passes over the animal as Eqn~\eqref{eq:fTt}.


\section{Web Appendix B: Constraint programming for enumerating all $\bm{\omega}^{(m)}$}
\label{appx:constrprog}

For efficient enumeration of the possible pairings within one segment, we define a simple constraint satisfaction problem (CSP)~\cite[Chapter 6]{russell-norvig-aima3}. A CSP is a triple \(\mathcal{P}=\langle \mathcal{X}, \mathcal{D}, \mathcal{C} \rangle\).  The CSP \(\mathcal{P}\) has a set of decision variables \(\mathcal{X}\), each of which has a set of possible values that it may take, called its \textit{domain}, where \(\mathcal{D}(x)\) is the domain of \(x \in \mathcal{X}\). In addition there is a set of constraints \(\mathcal{C}\) that restrict the combinations of values that may be taken by the variables. A constraint \(c\in \mathcal{C}\) is a relation defined on a set of variables: \(\mathrm{scope}(c)\subseteq \mathcal{X}\). A \textit{solution} is an assignment of values to variables such that each variable is assigned a value from its domain, and all constraints are satisfied.

We define a CSP for a segment as follows. Two detections by different observers may be paired if and only if the distance between them is less than or equal to \(d_{max}\). For each set \(\{i,j\}\) of two observations that may be paired, we define one decision variable \(x_{i,j}\) with domain \(\{0,1\}\). Variable \(x_{i,j}\) is equal to 1 in a solution if and only if the two observations are paired. \(\mathcal{X}\) is the set of all such decision variables \(x_{i,j}\). \(\mathcal{D}\) is the function \(\{ (x_{i,j}, \{0,1\})  \mid  x_{i,j} \in \mathcal{X} \}\).


Suppose we have two distinct sets, \(s_1=\{i,j\}\) and \(s_2=\{k,l\}\), where  \(i\) may be paired with \(j\), and \(k\) may be paired with \(l\), but the two sets are not disjoint: in other words \(s_1 \cap s_2 \ne \emptyset\). In all such cases we add the constraint \(c_{s_1,s_2} = (x_{i,j} = 0 \vee x_{k,l} = 0)\) to prevent such pairing. The set \(\mathcal{C}\) contains all such \(c_{s_1,s_2}\) and no other constraints.  All components of the CSP \(\mathcal{P}=\langle \mathcal{X},\mathcal{D},\mathcal{C} \rangle\) have now been defined.

We use a backtracking search procedure with forward checking~\cite[Chapter 6]{russell-norvig-aima3} to enumerate all solutions to the CSP. The set of solutions to the CSP corresponds one-to-one to the set of valid pairings within the segment. When a solution is found, the part of the likelihood pertaining to that pairing is calculated, avoiding the need to store the set of pairings and allowing efficient calculation of $\sum_{m_r=1}^{M_r}\mathcal{L}\left(\bm{\theta};\bm{s}^{(m_r)},\bm{\omega}^{(m_r)},\bm{t}^{(m_r)}\right)$.

\vspace{-1em}

\section{Web Appendix C: The relationship between $\sigma_{palm}$, $\sigma$ and mean animal speed}
\label{appx:sigmaspd}

The $\sigma$ of \cite{Stevenson+al:19}, which we call $\sigma_{palm}$ here, is based on the displacement of animals from the midpoint of their two locations after time $l$ has elapsed, which is normally distributed with mean zero and variance equal to $\sigma_{palm}^2$. If we let the signed distance between the first and second location be $Y$, then $Y/2\sim N(0,\sigma_{palm}^2)$ and hence $\sqrt{\{Y/(2\sigma_{palm})\}^2}=|Y|/(2\sigma_{palm})\sim\chi(1)$. Using the fact that the expected value of a $\chi(1)$ random variable is $\sqrt{2}/\Gamma(0.5)$, we have that $E\left\{|Y|/(2\sigma_{palm})\right\}=\sqrt{2}/\Gamma(0.5)$, and hence $2\sigma_{palm}=E(|Y|)\Gamma(0.5)/\sqrt{2}$. The distance $Y$ between the initial location and the location after $l$ seconds, of an animal following Brownian motion with rate parameter $\sigma$, has distribution $Y\sim N(0,\sigma^2l)$, so that $E\left\{|Y|/(\sigma\sqrt{l})\right\}=\sqrt{2}/\Gamma(0.5)$ and $\sigma\sqrt{l}=E(|Y|)\Gamma(0.5)/\sqrt{2}$, and hence $\sigma=2\sigma_{palm}/\sqrt{l}$. As the average speed of an animal over a period of $l$ seconds is $E(|Y|)/l$, the average speed over $l$ seconds of an animal following Brownian motion with rate parameter $\sigma$ can be written as $\sigma\sqrt{2}/\{\Gamma(0.5)\sqrt{l}\}$.


\section{Web Appendix C: Code to reproduce results of the paper}
\label{appx:code}
Source code for an \texttt{R} package called \texttt{LCE\_paper} to fit LCE models, and code to fit to the porpoise data and to conduct the simulations described in the paper, is available here: \verb|https://github.com/david-borchers/LCE_paper|. The package and code is also available at the Biometrics website on Wiley Online Library.

\bibliographystyle{biom}
\bibliography{dlb}

\end{document}